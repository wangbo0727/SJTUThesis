%# -*- coding: utf-8-unix -*-
%%==================================================
%% abstract.tex for SJTU Master Thesis
%%==================================================

\begin{abstract}

随着移动终端设备的广泛使用,移动在线广告正在占有更多的市场份额。按点击数量支付(CPC)和按展示时间支付(CPV)是现在在移动在线广告中最为流行的两种支付方式,他们分别把点击的次数和展示的时间作为支付报酬的依据。随着在线广告收益不断增长,点击欺诈也成为了一个严重的问题。点击欺诈是用非法的或不合理的点击来欺骗广告商,从而获得额外的收入。广告位置欺诈属于点击欺诈的一种,它是通过将广告摆放在移动应用中不合理的位置,来诱使用户在进行正常操作时失手去点击广告,从而产生一次点击的流量。之前关于点击欺诈的研究大多解决自动产生点击流量的广告欺诈,而不能用于解决这种引诱用户点击的广告牌位置欺诈的问题。在本文中,我们提出了一个基于众包及数据分析的移动应用广告位置欺诈检测系统。由于位置欺诈其本身的特性,其仅在用户正常使用应用时出现,利用众包的原理,用户使用每一个应用时,举报系统就可以对这个正在被使用的应用进行监测,这样便可以覆盖每一个可能的位置欺诈。我们将此系统部署在了10台平板电脑上,并且搜集了500多个移动应用来测试它的效果。实验结果显示我们的举报系统可以记录足够的信息来分析出哪个应用里有位置欺诈。除此之外,我们的系统还能找出一些特殊的位置欺诈,通过这种特殊情况的分析,我们认为这些特殊的欺诈无法被以往所有的自动分析机制发现。

\keywords{\large 点击欺诈 \quad 众包 \quad 移动应用}
\end{abstract}

\begin{englishabstract}

With the widespread use of mobile devices, mobile online advertising is taking more and more market share. Cost per click and cost per view are the most popular pricing modes in mobile internet advertising, which take effective clicks or displaying duration as the charging basis. However, at the same time, ad fraud, which use illegal and invalid click to fraud advertisers in order to obtain unreasonable income, become a serious problem. Most of the previous studies on click fraud in website focused on network traffic data analysis. This make them cannot solve the placement fraud problem, which use invalid placement to mislead user to click on it in mobile apps. In this paper, we propose a joint crowdsourcing and data analyzing based placement click fraud detection system. For the characteristic of placement fraud in mobile apps, automatic processing cannot cover every possible fraud. To overcome this, our report system provides a platform to find all possible placement fraud through crowdsourcing. Report system have three main services: a monitor service for monitoring user's call; a layout service for recording the screen; a data service for recording the backend data. Because the placement fraud only appears when users use the apps, the report system based on crowdsourcing can cover every possible placement fraud. We have implement our system in 10 tablets with 500 apps to evaluate its effectiveness. Experiment result shows that our approach can record enough data for analysing which app has placement fraud. What's more, our system can figure out some special placement fraud which pop ads when user is using other apps. This placement fraud cannot be solved through automatic method in previous studies.

\englishkeywords{\large SJTU, master thesis, XeTeX/LaTeX template}
\end{englishabstract}

