%# -*- coding: utf-8-unix -*-
%%==================================================
%% thesis.tex
%%==================================================

% 双面打印
\documentclass[doctor, fontset=adobe, openright, twoside]{sjtuthesis}
% \documentclass[bachelor, fontset=adobe, openany, oneside, submit]{sjtuthesis}
% \documentclass[master, fontset=adobe, review]{sjtuthesis}
% \documentclass[%
%   bachelor|master|doctor,	% 必选项
%   fontset=adobe|windows,  	% 只测试了adobe
%   oneside|twoside,		% 单面打印,双面打印(奇偶页交换页边距,默认)
%   openany|openright, 		% 可以在奇数或者偶数页开新章|只在奇数页开新章(默认)
%   zihao=-4|5,, 		% 正文字号:小四、五号(默认)
%   review,	 		% 盲审论文,隐去作者姓名、学号、导师姓名、致谢、发表论文和参与的项目
%   submit			% 定稿提交的论文,插入签名扫描版的原创性声明、授权声明 
% ]

% 逐个导入参考文献数据库
\addbibresource{bib/thesis.bib}
% \addbibresource{bib/chap2.bib}

\begin{document}

%% 无编号内容:中英文论文封面、授权页
%# -*- coding: utf-8-unix -*-
\title{上海交通大学学位论文 \LaTeX 模板示例文档}
\author{王\quad{}博}
\advisor{吴帆教授}
% \coadvisor{某某教授}
\defenddate{2017年12月3日}
\school{上海交通大学}
\institute{计算机科学技术与工程系}
\studentnumber{115033910028}
\major{计算机科学与技术专业}

\englishtitle{A Sample Document for \LaTeX-basedd SJTU Thesis Template}
\englishauthor{\textsc{Bo Wang}}
\englishadvisor{Prof. \textsc{Fan Wu}}
% \englishcoadvisor{Prof. \textsc{Uom Uom}}
\englishschool{Shanghai Jiao Tong University}
\englishinstitute{\textsc{Depart of Computer Science and Engineering %, School of Electronic Information and Electrical Engineering} \\
  \textsc{Shanghai Jiao Tong University} \\
  \textsc{Shanghai, P.R.China}}
\englishmajor{Computer Science and Technology Major}
\englishdate{Dec. 3rd, 2017}


\maketitle

\makeenglishtitle

\makeatletter
\ifsjtu@submit\relax
	\includepdf{pdf/original.pdf}
	\cleardoublepage
	\includepdf{pdf/authorization.pdf}
	\cleardoublepage
\else
\ifsjtu@review\relax
% exclude the original claim and authorization
\else
	\makeDeclareOriginal
	\makeDeclareAuthorization
\fi
\fi
\makeatother


\frontmatter 	% 使用罗马数字对前言编号

%% 摘要
\pagestyle{main}
%# -*- coding: utf-8-unix -*-
%%==================================================
%% abstract.tex for SJTU Master Thesis
%%==================================================

\begin{abstract}

随着移动终端设备的广泛使用,移动在线广告正在占有更多的市场份额。按点击数量支付(CPC)和按展示时间支付(CPV)是现在在移动在线广告中最为流行的两种支付方式,他们分别把点击的次数和展示的时间作为支付报酬的依据。随着在线广告收益不断增长,点击欺诈也成为了一个严重的问题。点击欺诈是用非法的或不合理的点击来欺骗广告商,从而获得额外的收入。广告位置欺诈属于点击欺诈的一种,它是通过将广告摆放在移动应用中不合理的位置,来诱使用户在进行正常操作时失手去点击广告,从而产生一次点击的流量。之前关于点击欺诈的研究大多解决自动产生点击流量的广告欺诈,而不能用于解决这种引诱用户点击的广告牌位置欺诈的问题。在本文中,我们提出了一个基于众包及数据分析的移动应用广告位置欺诈检测系统。由于位置欺诈其本身的特性,其仅在用户正常使用应用时出现,利用众包的原理,用户使用每一个应用时,举报系统就可以对这个正在被使用的应用进行监测,这样便可以覆盖每一个可能的位置欺诈。我们将此系统部署在了10台平板电脑上,并且搜集了500多个移动应用来测试它的效果。实验结果显示我们的举报系统可以记录足够的信息来分析出哪个应用里有位置欺诈。除此之外,我们的系统还能找出一些特殊的位置欺诈,通过这种特殊情况的分析,我们认为这些特殊的欺诈无法被以往所有的自动分析机制发现。

\keywords{\large 点击欺诈 \quad 众包 \quad 移动应用}
\end{abstract}

\begin{englishabstract}

With the widespread use of mobile devices, mobile online advertising is taking more and more market share. Cost per click and cost per view are the most popular pricing modes in mobile internet advertising, which take effective clicks or displaying duration as the charging basis. However, at the same time, ad fraud, which use illegal and invalid click to fraud advertisers in order to obtain unreasonable income, become a serious problem. Most of the previous studies on click fraud in website focused on network traffic data analysis. This make them cannot solve the placement fraud problem, which use invalid placement to mislead user to click on it in mobile apps. In this paper, we propose a joint crowdsourcing and data analyzing based placement click fraud detection system. For the characteristic of placement fraud in mobile apps, automatic processing cannot cover every possible fraud. To overcome this, our report system provides a platform to find all possible placement fraud through crowdsourcing. Report system have three main services: a monitor service for monitoring user's call; a layout service for recording the screen; a data service for recording the backend data. Because the placement fraud only appears when users use the apps, the report system based on crowdsourcing can cover every possible placement fraud. We have implement our system in 10 tablets with 500 apps to evaluate its effectiveness. Experiment result shows that our approach can record enough data for analysing which app has placement fraud. What's more, our system can figure out some special placement fraud which pop ads when user is using other apps. This placement fraud cannot be solved through automatic method in previous studies.

\englishkeywords{\large SJTU, master thesis, XeTeX/LaTeX template}
\end{englishabstract}



%% 目录、插图目录、表格目录
\tableofcontents
\listoffigures
\addcontentsline{toc}{chapter}{\listfigurename} %将插图目录加入全文目录
\listoftables
\addcontentsline{toc}{chapter}{\listtablename}  %将表格目录加入全文目录
\listofalgorithms
\addcontentsline{toc}{chapter}{算法索引}        %将算法目录加入全文目录

\include{tex/symbol} % 主要符号、缩略词对照表

\mainmatter	% 使用阿拉伯数字对正文编号

%% 正文内容
\pagestyle{main}
\chapter{绪论}
\label{chap:intro}
\section{研究背景及意义}
在线广告是一种利用网络作为途径来将广告内容分发展示给网络用户的广告形式,其在网络上的体现形式有很多种:邮件广告、搜索引擎广告、社交媒体广告、门户网站广告、移动应用广告等多种不同的形式。相对于传统的电视和报刊的广告形式,网络广告有着十分大的优势:网络广告的成本更低,投送方式更加简单,能够以较小的成本使广告大面积传播。自从第一个网络广告在1994年出现在美国以来,随着近几年互联网科技的发展,2011年网络广告在美国的收入已经超过有线电视\cite{Iab2012},并且在2016年达到了725亿美元\cite{Iab2016}。2017年的全球网络广告收入预计将达到2278亿美元。在线广告如此巨量的收益,使其逐渐成为了互联网经济中主要的经济收入来源,根据财报显示,社交媒体脸书公司营收的$90\%$ 以上都来自在线广告收入,而谷歌公司的这个比例更是高达$95\%$ 以上。随着近几年智能移动设备的普及,网络广告正在逐步的向移动设备上发展,移动广告就是把网络广告推送到智能手机或者平板这类移动终端上。随着现在智能设备的性能越来越高,屏幕的分辨率越来越大,移动广告可展示的内容也越来越多,其发展速度也越来越快。根据脸书的报告,移动广告收入已经占了其2016年第四季度收入的$84\%$。根据数据调查的预计,全球移动终端上的网络广告收益将在2018年超过桌面设备上的收益,而且份额会不断扩大。

在线广告投放系统能够较为有效地联系起广告主方和广告发布方(又可称为媒体方或流量方等),广告主通过向广告发布方支付一定的费用,使其广告能够在发布方的特定位置上展示广告。随着智能设备的普及,越来越多的广告主和广告发布商参与到在线广告的交易中,大量的参与者使这个投放过程变得复杂而且难以操作,因此,为了能够方便的服务广告主和广告发布商,能使资源进行合理的调配,就诞生了在线广告交易平台,即Ad Exchange。在线广告交易平台会通过拍卖的方式,将其所拥有的广告发布商的资源拍卖出去。与此同时,广告发布商会根据一定的规则向广告交易平台收取费用。这样尽管在交易中多了一个参与方,但是却能降低广告商和广告发布商的操作难度,同时优化了广告资源的调配。在线广告中,广告发布商向广告交易平台收费的主要规则为按点击次数收费(CPC)。但是,研究显示,$43\%$的用户点击行为都属于欺诈或者误击\cite{truth},也就是说,对这种对广告传播妹有任何意义点击,广告交易平台也需要向广告发布商支付这部分的费用。而且,为了避免损失,这部分费用最终会转嫁给广告商。这就意味着广告商浪费了大量的资金在一些没有产生效益的广告点击上,极大的损害了广告商的利益。近期的几项研究指出,移动设备(如智能手机和平板电脑)应用中的互联网广告也受到不同类型的欺诈行为的困扰。在2013年,由于点击欺诈问题,移动应用广告商损失了近10亿美元,占移动广告全部预算的$12\%$\cite{Bots}。

因为广告交易平台根据每千次展示或点击的数量作为定价方式来支付费用\cite{metwally2005duplicate} ,移动应用程序发布商也有动机进行广告欺诈行为:显而易见的更多的点击和展示将会给应用程序发布商带来更多的收益。对于由自动程序生成点击的网页广告点击欺诈行为,已经有过一些针对性的研究\cite{blizard2012click}\cite{miller2011s},但移动应用中的点击欺诈问题在近期才有一些关注,但主要针对广告位置欺诈检测的研究仍然不多。位置欺诈是通过在移动应用中不合理的位置摆放来诱使用户去点击广告来赚取点击的报酬。这种点击流量和通过程序自动产生的点击行为不同,其流量特征不易发现,很难通过流量分析和监控的方式去判断这一段点击流量是否存在欺诈。而且由于其性质的特殊,这种无效的点击有密度小、分散广的特点,其夹杂在有效的点击中,所以很难在在线广告交易平台一方进行分析处理。

\section{研究内容}

现有的针对移动应用中的点击欺诈问题的研究并不多,大多数研究者仍视其和一般网页广告有着相同的模式。但是实际上,对于移动应用中的广告位置欺诈问题,和一般的广告欺诈问题有非常不同的地方。相对于一般的网页广告欺诈,其特点包括以下几点:
\begin{itemize}
    \item 位置欺诈其广告内容合法,但是广告的放置违反了平台要求,致使用户在进行正常操作时,在不想点击广告的情况下,产生一次失误点击。这种点击数据和自动生成的点击数据流不同,其无效的点击分布不集中,夹杂在用户正常的点击中,不能通过流量分析来确定当前这个点击是否存在欺诈问题。
    \item 由于用户在使用应用时,应用界面会不停变换,这也导致了有位置欺诈的广告出现的时间的不确定性,以及出现位置的不确定性。而且很多应用的场景界面变换很多,这就使得利用程序自动分析变得比较困难。
\end{itemize}

考虑到以上的两个问题,为了解决位置欺诈,我们面临着两个挑战:
\begin{itemize}
    \item 与自动程序驱动的欺诈不同,位置欺诈会操纵广告的视觉布局,以便在真正的用户使用该应用程序时触发广告展示和无意的点击。 由于点击行为是由真正的用户完成的,所以该通过什么方法去判断这个点击是否是由位置欺诈引起。
    \item 广告位可能会在用户使用应用程序时随时出现,当通过集中自动处理,很难在广告位出现时抓住每一次欺诈。因此,如何才能尽量多的检测到每种欺诈的情况也是需要解决的问题。
\end{itemize}

为了解决这些挑战,我们提出了一个基于众包的报告系统,其通过众包来举报和检测位置欺诈。根据移动应用程序位置欺诈的特点,举报系统可以对当前的应用程序进行屏幕截取,并通过截屏来确定的位置欺诈位置。同时,系统可以收集数据包,并将数据和截图打包上传到服务器,之后对其进行分析,以获取具有位置欺诈的应用程序的信息。我们将举报系统安装在10台平板电脑上,每台平板都测试了超过500种移动应用,然后对收集的数据进行分析。结果表明,我们的系统是有效的,并且在指出含有一般位置欺诈的应用的同时,还找出了以前研究中从未提到的一中特殊的位置欺诈。

在本文中,我们的贡献总结如下:
\begin{enumerate}
    \item 与以往关于自动检测移动应用程序中的点击欺诈的研究相比,我们的系统基于众包思想,可以覆盖几乎每个的位置欺诈。
    \item 我们实现了举报系统并评估其性能,从而充分证明其在现实环境中的有效性。
    \item 我们分析从实验中收集的数据,结果显示了我们系统的效果并且分析了一些包括位置欺诈的应用程序的一些特性。
    \item 通过对实验异常数据的分析,我们发现了一种无法被集中处理分析的方法发现的特殊的位置欺诈模式。
\end{enumerate}

\section{论文结构}

文的其余部分组织如下:第一节将提供一些背景,动机和挑战的相关内容。在第二节中,我们将介绍举报系统的设计。第三节中将介绍系统的实现细节。在第四节中,我们将对举报系统的效果进行评估。在第五节中,给出了关于我们工作的结论。


\include{tex/intro}
\include{tex/example}
\include{tex/faq}
\include{tex/summary}

\appendix	% 使用英文字母对附录编号,重新定义附录中的公式、图图表编号样式
\renewcommand\theequation{\Alph{chapter}--\arabic{equation}}	
\renewcommand\thefigure{\Alph{chapter}--\arabic{figure}}
\renewcommand\thetable{\Alph{chapter}--\arabic{table}}
\renewcommand\thealgorithm{\Alph{chapter}--\arabic{algorithm}}

%% 附录内容,本科学位论文可以用翻译的文献替代。
\include{tex/app_setup}
\include{tex/app_eq}
\include{tex/app_cjk}
\include{tex/app_log}

\backmatter	% 文后无编号部分 

%% 参考资料
\printbibliography[heading=bibintoc]

%% 致谢、发表论文、申请专利、参与项目、简历
%% 用于盲审的论文需隐去致谢、发表论文、申请专利、参与的项目
\makeatletter

%%
% "研究生学位论文送盲审印刷格式的统一要求"
% http://www.gs.sjtu.edu.cn/inform/3/2015/20151120_123928_738.htm

% 盲审删去删去致谢页
\ifsjtu@review\relax\else
  \include{tex/ack} 	  %% 致谢
\fi

\ifsjtu@bachelor
  % 学士学位论文要求在最后有一个英文大摘要,单独编页码
  \pagestyle{biglast}
  \include{tex/end_english_abstract}
\else
  % 盲审论文中,发表学术论文及参与科研情况等仅以第几作者注明即可,不要出现作者或他人姓名
  \ifsjtu@review\relax
    \include{tex/pubreview}
    \include{tex/projectsreview}  
  \else
    \include{tex/pub}	      %% 发表论文
    \include{tex/projects}  %% 参与的项目
  \fi
\fi

% \include{tex/patents}	  %% 申请专利
% \include{tex/resume}	  %% 个人简历

\makeatother

\end{document}
